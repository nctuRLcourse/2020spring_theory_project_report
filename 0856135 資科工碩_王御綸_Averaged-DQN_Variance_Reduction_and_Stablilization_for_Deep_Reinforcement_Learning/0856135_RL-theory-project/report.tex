\documentclass{article}

% if you need to pass options to natbib, use, e.g.:
%     \PassOptionsToPackage{numbers, compress}{natbib}
% before loading neurips_2020

% ready for submission
 %\usepackage{neurips_2020}

% to compile a preprint version, e.g., for submission to arXiv, add add the
% [preprint] option:
\usepackage[preprint]{neurips_2020}

% to compile a camera-ready version, add the [final] option, e.g.:
%\usepackage[final]{neurips_2020}

% to avoid loading the natbib package, add option nonatbib:
%\usepackage[nonatbib]{neurips_2020}

\usepackage[utf8]{inputenc} % allow utf-8 input
\usepackage[T1]{fontenc}    % use 8-bit T1 fonts
\usepackage{hyperref}       % hyperlinks
\usepackage{url}            % simple URL typesetting
\usepackage{booktabs}       % professional-quality tables
\usepackage{amsfonts}       % blackboard math symbols
\usepackage{nicefrac}       % compact symbols for 1/2, etc.
\usepackage{microtype}      % microtypography

% Added packages
\usepackage{amsmath,amssymb,algorithm,algorithmic}

\title{Averaged-DQN: Variance Reduction and Stabilization}

% The \author macro works with any number of authors. There are two commands
% used to separate the names and addresses of multiple authors: \And and \AND.
%
% Using \And between authors leaves it to LaTeX to determine where to break the
% lines. Using \AND forces a line break at that point. So, if LaTeX puts 3 of 4
% authors names on the first line, and the last on the second line, try using
% \AND instead of \And before the third author name.

\author{%
  Yulun Wang\\
  Department of Computer Science\\
  National Chiao Tung University\\
  \texttt{xxx@nctu.edu.tw} \\

}

\begin{document}

\maketitle

\section{Introduction}
\label{section:intro}

The recent Deep Q-Network algorithm (DQN) was the first to successfully combine Deep Neural Network (DNN), a powerful non-linear approximation technique, with the Q-learning algorithm. However, the max operation in Q-learning can lead to overestimation of state-action values in the presence of noise.
\par In this work, variance analysis that explains some of the DQN problems are done, and later on, a solution called Averaged-DQN is proposed to solve the overestimation problem. It extends DQN by taking Q-values learned from few iterations before and averaging those values to produce the current action-value estimate. In the end, also, showing better results in the experiments done on Arcade Lerning Enviroment (ALE).
\par Different from Double-DQN, which tackle the overestimation problem by replacing the positive bias into negative one, Averaged-DQN directly reduce the variance and get better results.
\par This work greatly increases the stability of Deep Q-learning algorithms, but only adds the computation time in linear growth. Also, only a few changes are made from DQN to Averaged-DQN, which means it's easy to implement even if one don't know the theory behind it. All of these features make the proposed method valuable.

\iffalse
Please provide a clear overview of the selected paper. You may want to discuss the following aspects:
\begin{itemize}
    \item The main research challenges tackled by the paper
    \item The high-level technical insights into the problem of interest
    \item The main contributions of the paper (compared to the prior works)
    \item Your personal perspective on the proposed method
\end{itemize}
\fi

\section{Problem Formulation}
\label{section:problem}
\begin{itemize}
    \item Notations
    \paragraph{Markov decision process(MDP)\\}
    Defined by the tuple $(S, D, A, \mathcal{R}, {P(s';s,a)})$ where: $S$ is a finite set of states, $D$ is the 
    starting state distribution, $A$ is a finite set of actions, $\mathcal{R}$ is a reward function $\mathcal{R}:S \times A \to [0, R]$, and ${P'(s';s,a)}$ are the transition probabilities, with ${P'(s';s,a)}$ giving the next-state distribution upon taking action $a$ in state $s$.
    
    \paragraph{Value\\}
    Given $ 0 \leqslant \gamma < 1 $, we define the value function for a given policy $\pi$ as:
    \[V_{\pi}(s) \equiv (1 - \gamma) E \bigg[\sum_{t=0}^{\infty}\gamma^{t}\mathcal{R}(s_{t},a_{t}) | \pi,s\bigg]\]
    $(1 - \gamma)$ is for normalization, so $V_{\pi}(s) \in [0,R]$
    
    
    \paragraph{State-action value \\}
    We define state-action value as:
    \[Q_{\pi}(s,a) \equiv (1 - \gamma) \mathcal{R}(s,a) + {\gamma}E_{s^{'}{\thicksim}P(s^{'};s,a)} [V_{pi}(s^{'})]\]
    $Q_{\pi}(s,a) \in [0,R]$ due to normalization.
    
    
    \paragraph{Advantage}
    \[A_{\pi}(s,a) \equiv Q_{\pi}(s,a) - V_{\pi}(s)\]
    $A_{\pi}(s,a) \in [-R, R]$ due to normalization.
    
    
    \paragraph{Restart distribution \\} Draws the next state from the distribution $\mu$.\\
    
    
    \paragraph{$\gamma$-discount future state distribution \\} $\gamma$-discount future state distribution for
    a starting state distribution $\mu$:
    \[d_{\pi,\mu}(s) \equiv (1 - \gamma) \sum_{t=0}^{\infty} \gamma^{t} Pr(st=s;\pi, \mu) \] \\
    
    
    
    \item The optimization problem of interest
    \paragraph{}
    The goal of algorithm is to maximize the discounted reward from the start distribution $D$,
    \[\eta_{D} \equiv E_{s{\thicksim}D}[V_{\pi}(s)]\]
    The optimal policy exists which can maximize $V_{\pi}(s)$ for all states. \\
    
    \newpage
    \item The technical assumptions
    \paragraph{Restart distribution \\}
    The author desire an algorithm which uses only the MDP M and assumes access
    to restart distribution $\mu$.
    \paragraph{Approximate greedy policy chooser \\} 
    Crudely, the greedy policy chooser outputs a policy that usually chooses actions with largest state-action values of the current policy, ie it output
    an "approximate" optimal policy.
    
    \paragraph{Goal of the agent \\}
    Only consider the case where maximize the $\gamma$-discounted average reward
    from the start distribution D.
    
    
    
\end{itemize}


\section{Theoretical Analysis}
\label{section:analysis}

\subsection{Ensemble-DQN}
Following the Ensemble-DQN algorithm mentioned above:
\begin{align}
Q^E_i(s_0,a)&=\frac{1}{K}\sum^K_{k=1}Q(s_0,a,\theta^k_i)\\
            &=\frac{1}{K}\sum^K_{k=1}(Z^{k,i}_{s_{0},a}+y^i_{s_{0},a})\\
            &=\frac{1}{K}\sum^K_{k=1}Z^{k,i}_{s_{0},a}+y^i_{s_{0},a}\\
            &=\frac{1}{K}\sum^K_{k=1}Z^{k,i}_{s_{0},a}+\gamma Q^E_{i-1}(s_1,a)\\
            &=\frac{1}{K}\sum^K_{k=1}Z^{k,i}_{s_{0},a}+\frac{\gamma}{K}\sum^K_{k=1}Z^{k,i-1}_{s_{1},a}+\gamma y^{i-1}_{s_{2},a}
\end{align}
We can get (4) from (3) since we assume that the reward is 0 in everywhere. In addition, \begin{math}y^{j}_{s_{N-1},a}=0\end{math} in terminal states. By iteratively expanding \begin{math}y^{i-1}_{s_{2},a}\end{math}, we can get:
\[
Q^E_i(s_0,a)=\sum^{N-1}_{n=0}\gamma ^n\frac{1}{K}\sum^K_{k=1}Z^{k,i-n}_{s_{n},a}
\]
We also assume that TAEs are uncorrelated, meaning that \begin{math}Var(\sum X_i)=\sum Var(X_i)\end{math}. In the end, given \begin{math}Var[X]=E[X^{2}]+E[X]^2\end{math} and \begin{math}E[Z^{k,i}_{s,a}]=0\end{math}:
\[
Var[Q^E_{i}(s_{0},a)]=\sum^{N-1}_{n=0}\frac{1}{K}\gamma ^{2n}\sigma^2_{s_n}
\]
\subsection{Averaged-DQN}
Following the Averaged-DQN algorithm mentioned above:
\begin{align}
Q^E_i(s_0,a)&=\frac{1}{K}\sum^K_{k=1}Q(s_0,a,\theta_{i+1-k})\\
            &=\frac{1}{K}\sum^K_{k=1}(Z^{i+1-k}_{s_{0},a}+y^{i+1-k}_{s_{0},a})\\
            &=\frac{1}{K}\sum^K_{k=1}Z^{i+1-k}_{s_{0},a}+\frac{\gamma}{K}\sum^K_{k=1}Q^A_{i-k}(s_{1},a)\\
            &=\frac{1}{K}\sum^K_{k=1}Z^{i+1-k}_{s_{0},a}+\frac{\gamma}{K^2}\sum^K_{k=1}\sum^K_{k^{'}=1}Q(s_{1},a,\theta_{i+1-k-k^{'}})
\end{align}
Similar as before, assume reward is 0 everywhere, then we get (8) from (7). Also,  \begin{math}y^{j}_{s_{N-1},a}=0\end{math} in terminal states. By iteratively expanding
\begin{math}Q(s_{1},a,\theta_{i+1-k-k^{'}})\end{math}, we can get:
\begin{equation}
\begin{aligned}
Q^A_{i}(s_{0},a) = \frac{1}{K}\sum^K_{k_{1}=1}Z^{i+1-k_{1}}_{s_{0},a}
+\frac{\gamma}{K^2}\sum^K_{k_{1}=1}\sum^K_{k_{2}=1}Z^{i+1-k_{1}-k_{2}}_{s_{1},a}
+...\\
+\frac{\gamma^{N-1}}{K^N}\sum^K_{k_{1}=1}\sum^K_{k_{2}=1}...\sum^K_{k_{N}=1}Z^{i+1-k_{1}-k_{2}-...-k_{N}}_{s_{M-1},a}
\end{aligned}
\end{equation}
To calculate \begin{math}Var[Q^A_{i}(s_{0},a)]\end{math} easier, we'll compute the variance separately in equation (10) and sum them up. Note that assumptions like \begin{math}Var(\sum X_i)=\sum Var(X_i)\end{math} and \begin{math}Var[X]=E[X^{2}]+E[X]^2\end{math} will also be used in here.
\par For \begin{math}C\in \{1,2,...,N\}\end{math}, denote:
\begin{equation}
V_C = Var[\frac{1}{K^C}\sum^K_{k_{1}=1}\sum^K_{k_{2}=1}...\sum^K_{k_{C}=1}Z_{k_{1}+k_{2}+...+k_{C}}]
\end{equation}
where \begin{math}Z_C,Z_{C+1},...,Z_{K*C}\end{math} are independent and identically distributed TAE random variables, with \begin{math}E[Z_{i}]=0\end{math} and \begin{math}Var[Z_{i}]=\sigma^2_z\end{math}. Then:
\[
\begin{aligned}
V_C &=\frac{1}{K^{2C}}E[(\sum^{KC}_{j=C}n^C_{j}Z_{j})^{2}]\\
    &=\frac{\sigma^2_z}{K^{2C}}\sum^{KC}_{j=C}(n^C_{j})^2
\end{aligned}
\]
where \begin{math}n^C_{j}\end{math} denotes how many times \begin{math}Z_{j}\end{math} is counted in equation (11), and this can be found by converting the question to the number of solutions of the following equation:
\begin{equation}
k_{1}+k_{2}+...+k_{C}=j
\end{equation}
for \begin{math}k_{1},k_{2},...,k_{C}\in\{1,2,...,K\}\end{math}. After defining the new question,  \begin{math}n^C_{j}\end{math} can be written recursively:
\[
n^C_{j}=\sum^K_{i=1}n^{C-1}_{j-i}
\]
To be more precisely, consider equation (12) as distributing j same things into C same groups. Now, i items are given to a group, then the solution of the rest is \begin{math}n^C-1_{j-i}\end{math}. In the end, sum up the numbers of solution from \begin{math}i=1\end{math} to \begin{math}i=K\end{math}, and the answer will be \begin{math}n^C_{j}\end{math}.
\par Since the goal of this calculation is to bound the variance reduction coefficient, we will calculate the solution in the frequency domain, in which the bound can be easily obtained. Denote
\[
u^K_{j}=
\begin{cases}
    1,& \text{if } j\in\{1,2,...,K\}\\
    0,& \text{otherwise}
\end{cases}
\]
Trivially, \begin{math}n^C_{1}\end{math} will equal to \begin{math}u^K_{j}\end{math} for any \begin{math}j\in\mathbb{Z}\end{math} when \begin{math}C=1\end{math}, and \begin{math}n^C_{j}\end{math} can be written  recursively as:
\[
\begin{aligned}
n^C_{j} &=\sum^\infty_{i=-\infty}n^{C-1}_{j-i}\cdot u^K_{i}\\
        &\equiv(n^{C-1}_{j-i}\circledast u^K)_j\\
        &=(u^K\circledast u^K...\circledast u^K)_j
\end{aligned}
\]
where \begin{math}\circledast\end{math} is the discrete convolution.
\par Next, denote the Discrete Fourier Transform (DFT) of \begin{math}u^K=(u^K_{m})^{M-1}_{m=0}\end{math} as \begin{math}U=(U_{m})^{M-1}_{m=0}\end{math}, and by using Parseval’s theorem, we have:
\[
\begin{aligned}
V_C &=\frac{\sigma^2_z}{K^{2C}}\sum^{M-1}_{m=0}|u^K\circledast u^K...\circledast u^K)_{m}|^2\\
    &=\frac{\sigma^2_z}{K^{2C}}\frac{1}{M}\sum^{M-1}_{m=0}|U_{m}|^{2C}
\end{aligned}
\]
where N is the length of the vectors u and U and is taken large enough so that the sum includes all nonzero elements of the convolution. Finally, we can sum up all of $V_C$:
\[
\begin{aligned}
Var[Q^A_{i}(s_{0},a)]   &=\sum^N_{n=1}V_n\gamma^{2(n-1)}\\
                        &=\sum^N_{n=1}\frac{1}{K^{2n}}\frac{1}{M}\sum^{M-1}_{m=0}|U_{m}|^{2n}\gamma^{2(n-1)}\sigma^2_{s_{m}}
\end{aligned}
\]
\subsection{Coefficient Bounding Analysis}
First, let's calculate variance of original DQN, which is similar with Ensemble-DQN:
\[
\begin{aligned}
Q^{DQN}(s_{0},a,\theta_{i}) &=Z^i_{s_{0},a}+y^i_{s_{0},a}\\
                            &=Z^i_{s_{0},a}+\gamma Q(s_{1},a,\theta_{i-1})\\
                            &=Z^i_{s_{0},a}+\gamma(Z^{i-1}_{s_{1},a}+y^{i-1}_{s_{1},a})
\end{aligned}
\]
And the variance will be:
\[
Var[Q^{DQN}(s_{0},a,\theta_{i})]=\sum^{N-1}_{n=0}\gamma^{2n}\sigma^2_{s_{m}}
\]
According to the results from above, 
\[
\begin{aligned}
&Var[Q^E_{i}(s_{0},a)]=\sum^{N-1}_{n=0}\frac{1}{K}\gamma ^{2n}\sigma^2_{s_n}\\
&Var[Q^A_{i}(s_{0},a)]=\sum^N_{n=1}\frac{1}{K^{2n}}\frac{1}{M}\sum^{M-1}_{m=0}|U_{m}|^{2n}\gamma^{2(n-1)}\sigma^2_{s_{m}}
\end{aligned}
\]
We can easily find out the difference between Ensemble-DQN and DQN:
\newtheorem{prop1}{Proposition}
\begin{prop1}
Variance of Ensemble-DQN is K times smaller than DQN
\end{prop1}
As for Averaged-DQN, let's first do some change to the equation:
\[
\begin{aligned}
Var[Q^A_{i}(s_{0},a)]   &=\sum^N_{n=1}\frac{1}{K^{2n}}\frac{1}{M}\sum^{M-1}_{m=0}|U_{m}|^{2n}\gamma^{2(n-1)}\sigma^2_{s_{m}}\\
                        &=\sum^{N-1}_{n=0}\frac{1}{K^{2(n+1)}}\frac{1}{M}\sum^{M-1}_{m=0}|U_{m}|^{2(n+1)}\gamma^{2n}\sigma^2_{s_{m}}\\
\end{aligned}
\]
then analyze the difference between Averaged-DQN and DQN by using Parseval’s theorem, and the facts that \begin{math}\frac{1}{K}|U_{m}|\leq 1\end{math}, \begin{math}\frac{1}{K}|U_{m}|=1 \text{ only if } n=0\end{math}:
\[
\begin{aligned}
\frac{1}{K^{2(n+1)}}\frac{1}{M}\sum^{M-1}_{m=0}|U_{m}|^{2(n+1)}&=\frac{1}{M}\sum^{M-1}_{m=0}|U_{m}/K|^{2(m+1)}\\
&<\frac{1}{M}\sum^{M-1}_{m=0}|U_{m}/K|^{2}\\
&=\frac{1}{K^2}\sum^{M-1}_{m=0}|u^K_{m}|^2\\
&=1/K
\end{aligned}
\]
And this gives the following conclusion:
\newtheorem{prop2}[prop1]{Proposition}
\begin{prop2}
Variance of Averaged-DQN is at lease K times smaller than DQN, and even smaller than Ensemble-DQN
\end{prop2}

\iffalse
Please present the theoretical analysis in this section. Moreover, please formally state the major theoretical results using theorem/proposition/corollary/lemma environments. Also, please clearly highlight your new proofs or extensions (if any).
\fi



\section{Conclusion}
\label{section:conclusion}
Please provide succinct concluding remarks for your report. You may discuss the following aspects:
\begin{itemize}
    \item The potential future research directions \\
    This paper prove the Conservative Policy Iteration can find the optimal policy after reasonable update times, the future research direction can be applying the algorithm in the existing reinforcement learning environment (ex.gym in python) to evaluate the performance of the algorithm. It can be compared with the traditional value-iteration, policy iteration algorithm for solving the game like Taxi-v2, FrozenLake8x8, etc. \\
    And a MDP environment like Figure3.2 can be created, with this environment, Conservative Policy Iteration
    can compare its performance with policy gradient method to see if the problem of unbalance updating on two states can be reduced by the introducing of restart distribution and conservative policy update rule (by comparing $\rho(i)$ and $\rho(j)$). \\
    \item Technical limitations \\
    In this algorithm, in the step 2 of Conservative Policy Iteration, to estimate $A_{\pi, \mu}(\pi')$, we 
    need to compute the expectation value of $A_{\pi}(s,a)$, but if the state number is very large or the action is continuous, the program will take unreasonable computation time and memory. This could be solved by applying "sampling" to estimate the true expectation value (ex. Monte Carlo method). \\
    \item Latest results on the problem of interest \\
    Trust Region Policy Optimization (TRPO) [\cite{DBLP:journals/corr/SchulmanLMJA15}] is also an algorithm that can optimize the policy and which is effective for finding large nonlinear policies. And the theorem 4.1 of this paper is also a theoretical basis for the prove of TRPO. 
    And compare to TRPO, Proximal Policy Optimization (PPO) 
    [\cite{DBLP:journals/corr/SchulmanWDRK17}] is another algorithm in
    the policy gradient family which is much simpler to implement, and enable the
    multiple epochs of minibatch multiple epochs of minibatch policy gradient updates.
    
    
\end{itemize}

{
\small
\bibliographystyle{unsrtnat}
%Bibliographic references
\begin{thebibliography}{9}
\bibitem{latexcompanion} 
Oron Anschel, Nir Baram, and Nahum Shimkin.
Averaged-DQN: Variance Reduction and Stabilization for Deep Reinforcement Learning, ICML 2017.

\bibitem{einstein} 
Mnih, Volodymyr, Kavukcuoglu, Koray, Silver, David,
Graves, Alex, Antonoglou, Ioannis, Wierstra, Daan, and
Riedmiller, Martin.
Playing Atari with deep reinforcement learning. arXiv preprint arXiv:1312.5602, 2013.

\bibitem{knuthwebsite} 
Thrun, Sebastian and Schwartz, Anton.
Issues in using
function approximation for reinforcement learning. In
Proceedings of the 1993 Connectionist Models Summer
School Hillsdale, NJ. Lawrence Erlbaum, 1993.
\end{thebibliography}
}


\end{document}
